\documentclass[a4paper,10pt]{article}

 % --- GRUNDEINSTELLUNGEN ---
 \usepackage[ngerman]{babel}
 \usepackage{fontspec} % Erlaubt die Nutzung von System-Schriftarten
 \usepackage{microtype}
 \usepackage{geometry}
 \usepackage{graphicx}
 \usepackage[svgnames]{xcolor}
 \usepackage{tikz}
 \usepackage{fontawesome5}
 \usepackage{enumitem}
 \usepackage{hyperref}

 % --- SEITENLAYOUT & LINKS ---
 \geometry{left=1.5cm, right=1.5cm, top=1.5cm, bottom=1.5cm}
 \pagestyle{empty}
 \setlength{\parindent}{0pt}
 \hypersetup{colorlinks=true, urlcolor=primarycolor, linkcolor=primarycolor}

 % --- LADE DIE GEBÜNDELTE SCHRIFTART PER DATEINAME ---
 % Dies ist der entscheidende Teil. Wir sagen fontspec, wo genau die Dateien liegen.
% --- ALLERLETZTER VERSUCH: Pfad direkt in die Dateinamen integrieren ---
% --- SCHRIFTART & FARBEN ---
\setsansfont{Lato} % Lade die Schriftart "Lato" einfach nach Namen
\renewcommand{\familydefault}{\sfdefault}

 \definecolor{primarycolor}{HTML}{5B9BD5}
 \definecolor{graytext}{HTML}{595959}
 \definecolor{lightgray}{HTML}{D9D9D9}

 % ... (der Rest Ihrer Datei ab \newcommand{\VAR}[1]... bleibt gleich) ...

% --- PLATZHALTER-BEFEHL ---
\newcommand{\VAR}[1]{\texttt{#1}}

% --- HELFER-BEFEHLE FÜR DAS DESIGN ---
\newcommand{\sectiontitle}[1]{%
  {\Large\color{primarycolor}\textbf{#1}}\\[2mm]
  \textcolor{lightgray}{\rule{\linewidth}{1pt}}\\[4mm]
}
\newcommand{\skill}[1]{\item \textcolor{graytext}{#1}}
\setlist[itemize]{leftmargin=*, noitemsep, topsep=2pt, partopsep=0pt, parsep=2pt}

%%%%%%%%%%%%%%%%%%%%%%%%%%%%%%%%%%%%%%%%%%%%%%%%%%%%%%%%%%%%%%%%%%%%%%%%%%%%%%%%
% --- BEGINN DES DOKUMENTS ---
%%%%%%%%%%%%%%%%%%%%%%%%%%%%%%%%%%%%%%%%%%%%%%%%%%%%%%%%%%%%%%%%%%%%%%%%%%%%%%%%
\begin{document}

% --- KOPFZEILE: NAME UND TITEL ---
\begin{center}
    {\Huge\color{primarycolor}\textbf{\VAR{name}}}\\[2mm]
    {\large\color{graytext}\VAR{titel}}
\end{center}
\vspace{8mm}

% --- HAUPTBEREICH: ZWEISPALTIGES LAYOUT ---
\begin{minipage}[t]{0.32\textwidth} % Linke, schmalere Spalte
    \raggedright % VERBESSERUNG: Linksbündiger Flattersatz für bessere Lesbarkeit in schmalen Spalten

    % --- PROFILBILD (RUND) ---
    \begin{center}
        \begin{tikzpicture}
            \clip (0,0) circle (1.7cm);
            % HINWEIS FÜR JAVA: Ersetzen Sie \VAR{photo_path} mit dem Pfad zu der vom User hochgeladenen Bilddatei.
            \node {\includegraphics[width=3.4cm]{\VAR{photo_path}}};
        \end{tikzpicture}
    \end{center}
    \vspace{8mm}

    % --- KONTAKT ---
    \sectiontitle{Kontakt}
    \begin{tabular}{@{}p{1cm}p{6cm}}
        \faPhone & \textcolor{graytext}{\VAR{telefon}} \\
        \faEnvelope & \textcolor{graytext}{\VAR{email}} \\
        \faMapMarker & \textcolor{graytext}{\VAR{adresse}} \\
    \end{tabular}
    \vspace{8mm}

    % --- ÜBER MICH ---
    \sectiontitle{Über mich}
    \textcolor{graytext}{\VAR{ueber_mich_text}}
    \vspace{8mm}

    % --- KENNTNISSE ---
    \sectiontitle{Kenntnisse}
    \begin{itemize}
        % HINWEIS FÜR JAVA: Iterieren Sie durch die Liste der Skills aus dem JSON.
        % Erzeugen Sie für jeden Skill eine Zeile im Format "\skill{Skill-Name}"
        % und ersetzen Sie den gesamten Platzhalter \VAR{kenntnisse_block}
        % mit dem Ergebnis-String.
        \VAR{kenntnisse_block}
    \end{itemize}

\end{minipage}
\hspace{0.04\textwidth} % Abstand zwischen den Spalten
\begin{minipage}[t]{0.64\textwidth} % Rechte, breitere Spalte

    % --- AUSBILDUNG ---
    \sectiontitle{Ausbildung}
    % HINWEIS FÜR JAVA: Iterieren Sie durch die Ausbildungs-Liste im JSON.
    % Bauen Sie für jeden Eintrag einen Block und ersetzen Sie den Platzhalter.
    \VAR{ausbildung_block}

    % --- BERUFSERFAHRUNG ---
    \sectiontitle{Berufserfahrung}
    % HINWEIS FÜR JAVA: Iterieren Sie durch die Berufserfahrungs-Liste im JSON.
    % Bauen Sie für jeden Eintrag einen Block und ersetzen Sie den Platzhalter.
    \VAR{berufserfahrung_block}

    % --- PROJEKTERFAHRUNG ---
    \sectiontitle{Projekterfahrung}
    % HINWEIS FÜR JAVA: Iterieren Sie durch die Projekt-Liste im JSON.
    % Bauen Sie für jeden Eintrag einen Block (inkl. \href-Befehl für Links)
    % und ersetzen Sie diesen Platzhalter.
    \VAR{projekterfahrung_block}

\end{minipage}

\end{document}